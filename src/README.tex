\documentclass{article}
\usepackage{graphicx}
\usepackage{amssymb}
\begin{document}
    \title{Final Project: Points Analysis}
    \maketitle
    \section{Introduction and Motivation}
    \textbf{\checkmark Provide any information and context needed to understand the motivation for the project}

    Our final project is an analysis of previous points put towards predicting how many points would be necessary to get into future courses. As students at Colorado College, we understand the inherent difficulty determining the number of points necessary to be put into any given class. We decided to take on this project in order to investigate the possibility for students to more computationally predict the number of points necessary.

    \par
    \textbf{\checkmark Clearly describe the purpose of the project, and what functionality you intend to support (at a high level, save implementation details for later)}

    The purpose of our project is to provide with a way to predict the demand and minimum number of points required to get into any given class. The intention is that you select a class, and based on prior years' information available, you would be able to predict the number of points required to get the class the coming semester, assuming it is offered.

    \par
    \textbf{Summarize what you build (again, at a high level) and how it achieves the goals of your project}

    What we built is a computer-based GUI that you can select a class within, and the computer will determine how many points are necessary for you to get into said class. It also would display how many points were necessary in previous years to get into the class. To put it simply, this does achieve the goals of our project, and it does beyond a shadow of a doubt. This easily ticks the key prongs our project had intent to do and then some. Nonetheless, let's walk through them. First, we have taking in the information. Our GUI takes in information from the user. They are allowed to select a course from a list of courses previously offered within the database, including adjuncts and half blocks. Secondly, we have the prediction. Our computer is capable of taking information of prior classes and running its prediction algorithm in multiple languages. Third, we have displaying it. Our GUI shows not only the predicted number of points, but the number of points that were necessarily to get into the class previously.



    \section{Tools}
    We used Java to compile this, as well as Python for some compilation of settings. One such library that we used in Java was an expansion to FileReader, Alpache POI. It allowed us to analyze Excel documents, and from those
\end{document}